% !TEX TS-program = pdflatex
\documentclass{arteacle}
%use the option bib-apa to use apa-style referencing

% To change any options that are being overwritten by the class,
% You will have to put this in an \AtEndPreamble statement

\title{Non-mutating algorithms to choose $n$ unique elements at random from an array}
\author{Jake W. Ireland}

\algdef{SE}[SUBALG]{Indent}{EndIndent}{}{\algorithmicend\ }%
\algtext*{Indent}
\algtext*{EndIndent}


\begin{document}
		
%%%%%%%% QUESTION 1
\begin{algorithm}[h!]
		\begin{algorithmic}
			\caption{Takes parameters \texttt{inArr} and \texttt{n} (potentially high run time)}
			\State initialise \texttt{outArr} as empty array
			\While{$\texttt{outArr.length}\leq \texttt{n}$}
				\State \textbf{choose} random \texttt{elem} from \texttt{inArr}
				\If{\texttt{elem} not in \texttt{outArr}}
					\State \textbf{push} \texttt{elem} to \texttt{outArr}
				\EndIf
			\EndWhile
			\State \textbf{return} \texttt{outArr}
		\end{algorithmic}
\end{algorithm}

\begin{algorithm}[h!]
		\begin{algorithmic}
		\caption{Takes parameters \texttt{inArr} and \texttt{n} (potentially high memory usage/allocations)}
			\State initialise \texttt{outArr} as empty array
			\State \textbf{copy} \texttt{inArr} as \texttt{tempArr}
        	\State\textbf{do }\texttt{n}\textbf{ times}
        		\Indent
        			\State \textbf{take} random \texttt{elem} from \texttt{tempArr}
					\State \textbf{push} \texttt{elem} to \texttt{outArr}
        		\EndIndent
			\State \textbf{return} \texttt{outArr}
		\end{algorithmic}
\end{algorithm}
		
			
	
\end{document}
